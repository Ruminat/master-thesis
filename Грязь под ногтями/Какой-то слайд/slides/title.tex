%!TeX root = ../Asq.tex
\begin{frame}[plain]{}
  \setbeamertemplate{section in toc}[sections numbered]
  % \tableofcontents[hideallsubsections]
  % \begin{center}%
  %   \footnotesize
  %   Санкт-Петербургский политехнический университет Петра Великого \\ 
  %   Институт компьютерных наук и технологий \\ 
  %   Высшая школа интеллектуальных систем и суперкомпьютерных технологий
  % \end{center}
  % \vfill
  \begin{center}%
    % Выпускная квалификационная работа бакалавра \\
    \Large
    {Автоматическая обработка и генерация текста на естественных языках с применением искусственных нейронных сетей} \\[6pt]
    \scriptsize
    \textbf{Цель:} разработать систему, способную автоматически упрощать и реферировать тексты (предположительно на русском и японском языках). \\
    \textbf{Мотивация:} слабо развитые средства генерации текстов для русского и (в особенности) японского языков.
    Особенно актуальна тема для японского языка, так как многие иероглифы, используемые в сложных японских текстах школьники изучают в старших классах, упрощение таких текстов делает их доступными для большего числа людей, также это полезно для людей, изучающих этот язык.
  \end{center}
  \vfill
  \footnotesize
  \begin{tabularx}{\textwidth}{LcR}%
    Выполнил: &  &  Научный руководитель: \\ 
    студент гр.~3540203/00101 &  &  к.\,т.\,н., доцент ВШИСиСТ \\
    \textbf{Фурман Владислав Константинович} &  & \textbf{Пак Вадим Геннадьевич}
  \end{tabularx}%

  % \begin{center}%
  %   Санкт-Петербург \\ 
  %   2020\,г.
  % \end{center}
\end{frame}
