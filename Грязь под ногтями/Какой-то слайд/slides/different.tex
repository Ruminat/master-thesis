%!TeX root = ../Asq.tex
\begin{frame}[fragile]{Различные формулировки запроса}%
  Оба запроса:
  \begin{enumerate}%
    \item «\textit{Выведи фамилию, имя и зарплату сотрудников с зарплатой больше 10000 или именем = 'David', отсортируй по зарплате по убыванию и по имени}»;
    \item «\textit{Сортировка по заработку по убыванию и имени, причём зарплата выше 10000 или имя равно 'David', вывести нужно фамилию, имя и оклад}»;
  \end{enumerate}
  "--- дают одинаковый результат:

  \begin{minted}[tabsize=2, mathescape, fontsize=\footnotesize]{sql}
SELECT last_name, first_name, salary
FROM hr.employees
WHERE salary > 10000
  -- здесь неявно для пользователя используются TRIM и LOWER.
  OR TRIM(LOWER(first_name)) = TRIM(LOWER('David'))
ORDER BY salary DESC, first_name
  \end{minted}

\end{frame}
