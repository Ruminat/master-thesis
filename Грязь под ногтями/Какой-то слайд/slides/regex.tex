%!TeX root = ../Asq.tex
\begin{frame}[fragile]{Абстрактные регулярные выражения}%
  \begin{itemize}%
    \item Являются обобщением регулярных выражений;
    \item принцип работы основан на недетерминированных конечных автоматах;
    \item работают с объектами любой природы и любыми предикатами;
    \item выполняют поиск среди токенов, \textbf{сохраняя структуру шаблона};
    \item позволяют писать шаблоны в \textbf{декларативном стиле}.
  \end{itemize}

  \begin{figure}[H]%
    \small
    \centering
    Конечный автомат для регулярного выражения \texttt{table (connector table)*}
    \begin{tikzpicture}[->,>=stealth',shorten >=1pt,auto,node distance=4cm, semithick]
      \tikzstyle{every state}=[fill=none,draw=black,text=black]

      \node[initial,state]     (E)                    {$A$};
      \node[state]             (A) [right=1.6cm of E] {$B$};
      \node[state]             (B) [right=2.2cm of A] {$C$};
      \node[state]             (C) [right=1.6cm of B] {$D$};
      \node[accepting,state]   (D) [right=0.6cm of C] {$E$};

      \path (A) edge node {\texttt{connector}} (B)
            (B) edge node {\texttt{table}} (C)
            (A) edge [bend left] node {} (C)
            (C) edge [bend left] node {\texttt{connector}} (B)
            (C) edge node {} (D)
            (E) edge node {\texttt{table}} (A);
    \end{tikzpicture}
  \end{figure}
  Ему соответствует, к примеру, такой запрос: «\textit{сотрудники, отделы и страны}».
\end{frame}
