%!TeX root = ../My_thesis.tex


% .---|||___|||--- C H A P T E R ---|||___|||---. %
\chapter{Результаты и улучшение модели}
% .---|||___|||--- C H A P T E R ---|||___|||---. %


% .---|||___|||--- S E C T I O N ---|||___|||---. %
\section{Варианты модификации модели}
% .---|||___|||--- S E C T I O N ---|||___|||---. %


Берём да улучшаем!


% .---|||___|||--- S E C T I O N ---|||___|||---. %
\section{Результаты}
% .---|||___|||--- S E C T I O N ---|||___|||---. %


Вот такие результаты

\begin{table}[H]% Пример оформления таблицы
  \centering\small
  \caption{Метрики полученных моделей}
  \label{metrics}
    \begin{tabular}{|l|l|l|}
      \hline
      \textbf{Модель} & \textbf{BLEU} & \textbf{SARI} \\ \hline
      Transformer & 45{,}63 & 63{,}17 \\ \hline
      Pretrained Transformer & 50{,}78 & 67{,}33 \\ \hline
      Pretrained Encoder & 46{,}87 & 64{,}27 \\ \hline
    \end{tabular}
    \normalsize
\end{table}

\begin{itemize}%
  \item Transformer "--- изначальная модель;
  \item Pretrained Transformer "--- предобученная модель, которую дообучаем, замедляя обучение encoder'а;
  \item Pretrained Encoder "--- изначальная модель, в которой заменяем encoder на предобученный, замедляя его обучение.
\end{itemize}
