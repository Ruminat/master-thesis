%!TeX root = ../My_thesis.tex
\chapter*{Введение} % * не проставляет номер
\addcontentsline{toc}{chapter}{Введение} % вносим в содержание

С помощью естественного языка можно выразить любую мысль, любую идею.
Любое изображение или звук можно описать словами.
Текст является всеобъемлющим средством передачи информации.
Что означает, что обработка текстов на естественных языках (Natural Language Processing, NLP) является крайне важной и актуальной проблемой.

Существует большое множество задач в области обработки текстов, например:
\begin{itemize}%
  \item перевод с одного языка на другой (например, перевод с русского на японский или обратно);
  \item или же монолингвистический перевод (перевод с языка в него же),  как, например, упрощение текстов (понижение сложности слов, выражений, грамматики, сохраняя при этом исходный смысл текста);
  \item классификация текстов (положительный или отрицательный отзыв, фильтрация спама и~т.\,д.);
  \item генерация текстов (например, из заданного заголовка сгенерировать статью);
  \item реферирование текстов (из большого по объёму документа или набора документов выделить ёмкую основную мысль);
\end{itemize}

Для японского и китайского языков особый интерес представляет задача упрощения текстов, так как эти языки используют иероглифическую письменность, где для чтения текстов нужно знать чтение и значение отдельных иероглифов (в японском языке большинство иероглифов имеют несколько чтений, порой даже больше~10).
Это может значительно сузить круг возможных читателей какой-либо текста "--- дети изучают иероглифы, начиная с первого класса школы и до самого выпуска.
То же касается и иностранцев, имеющих довольно ограниченное знание иероглифов.
Причём даже взрослые японцы и китайцы могут испытывать трудности с иероглифами, особенно связанные с юридическими документами.
Количество иероглифов довольно высоко, в среднем, взрослый японец знает порядка 2\,000 иероглифов, взрослый китаец "--- 8\,000 (хотя самих иероглифов значительно больше "--- не менее 80\,000, "--- но большинство из них используются крайне редко).
Поэтому есть высокая потребность в упрощении текстов для увеличения количества их потенциальных читателей.

Более того, упрощение текстов может повысить эффективность других задач NLP, как, например, реферирование, извлечение информации, машинный перевод и~т.\,д.

Целью данной научно-исследовательской работы является окончательная программная реализация системы для упрощения предложений на японском языке.

Для достижения данной цели необходимо выполнить следующие задачи:
\begin{itemize}%
  \item описать архитектуру реализуемой системы, обзор инструментов для разработки;
  \item сделать обзор на архитектуру модели Transformer, как с теоритической, так и с практической точек зрения;
  \item описать реализацию системы упрощения (модель, сервер и приложение);
  \item провести демонстрацию примеров работы разработанной системы на предложениях как из корпуса, так и вне корпуса.
\end{itemize}

% Целью данной работы является разработка системы автоматического упрощения текстов на японском языке.

% Для достижения данной цели необходимо выполнить следующие задачи:
% \begin{itemize}%
%   \item рассмотреть проблемы обработки текстов на японском языке, а также существующие решения в области упрощения текстов;
%   \item исследовать различные архитектуры нейронных сетей;
%   \item разработать и реализовать описанную систему;
%   \item исследовать качество разработанного решения, а также его эффективность в улучшении других задач NLP.
% \end{itemize}
