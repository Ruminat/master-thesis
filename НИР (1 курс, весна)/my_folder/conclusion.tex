%!TeX root = ../My_thesis.tex
\chapter*{Заключение} \label{ch-conclusion}
\addcontentsline{toc}{chapter}{Заключение}	% в оглавление 

В данной научно-исследовательской работе были поставлены цель и задачи предстоящей магистерской дипломной работы, были рассмотрены следующие её аспекты:
\begin{itemize}%
  \item собенности обработки японского языка "--- в частности, японская письменность, морфология, грамматика, упрощение лексики с грамматикой с примерами;
  \item существующие решения и датасеты "--- были рассмотрены актуальные исследования в данной теме, а также датасеты, которые в них использовались, было решено разрабатывать систему, основанную на модели Transformer, так как она кажется наиболее перспективной;
  \item «где используется упрощение» "--- были рассмотрены 2 сервиса, в которых применяют упрощение, о котором говорилось в НИР: Simple English Wikipedia и News Web Easy, была также затронута тема текущего положения систем упрощения в целом.
\end{itemize}