%!TeX root = ../My_thesis.tex
%% Не менять - Do not modify
%%\thispagestyle{empty}%
%\setcounter{sumPageFirst}{\value{page}} %сохранили номер первой страницы Реферата
\ifnumequal{\value{sumPrint}}{1}{% если двухсторонняя печать Задания, то...
	\newgeometry{twoside,top=2cm,bottom=2cm,left=3cm,right=1cm,headsep=0cm,footskip=0cm}
	\savegeometry{MyTask} %save settings
	\makeatletter % задаём оформление второй страницы ВКР как нечетной, а третьей - как чётной
	\let\tmp\oddsidemargin
	\let\oddsidemargin\evensidemargin
	\let\evensidemargin\tmp
	\reversemarginpar
	\makeatother
}{} % 
\pagestyle{empty} % удаляем номер страницы на втором/третьем листе 
\chapter*[Count-me]{Реферат} % * - не нумеруем
\thispagestyle{empty}% удаляем параметры страницы
%\setcounter{sumPageFirst}{\value{page}}
%sumPageFirst \arabic{sumPageFirst}
%
%
%% Возможность проверить другие значения счетчиков - debugging
%\ref*{TotPages}~с.,
%\formbytotal{mytotalfigures}{рисун}{ок}{ка}{ков},
%\formbytotal{mytotaltables}{таблиц}{у}{ы}{},
%There are \TotalValue{mytotalfigures} figures in this document
%There are \TotalValue{mytotalfiguresInApp} figuresINAPP in this document
%There are \TotalValue{mytotaltables} tables in this document
%There are \TotalValue{mytotaltablesInApp} figuresINAPP in this document
%There are \TotalValue{myappendices} appendix chapters in this document
%\total{citenum}~библ. наименований.



%% Для того, чтобы значения счетчиков корректно отобразились, необходимо скомпилировать файл 2-3 раза
На \total{mypages}~c.,  
\formbytotal{myfigures}{рисун}{ок}{ка}{ков},
\formbytotal{mytables}{таблиц}{у}{ы}{},
\formbytotal{myappendices}{приложен}{ие}{ия}{ий}.  

%\noindent
{\MakeUppercase{Ключевые слова: \keywordsRu}.}
% \footnote{Всего \textbf{слов}: от 3 до 15. Всего \textbf{слов и словосочетаний}: от 3 до 5. Оформляются в именительном падеже множественного числа (или в единственном числе, если нет другой формы), оформленных по правилам русского языка. \textit{Внимание! Размещение сноски после точки является примером как запрещено оформлять сноски.}} % Ключевые слова из renames.tex

Тема выпускной квалификационной работы: <<\thesisTitle>>%
%\footnote{Реферат \textbf{должен содержать}: предмет, тему, цель ВКР; метод или методологию проведения ВКР: результаты ВКР: область применения результатов ВКР; выводы.}
.


\abstractRu%
%\footnote{ОТ 1000 ДО 1500 печатных знаков (ГОСТ Р 7.0.99-2018 СИБИД) на русский или английский текст. Текст реферата повторён дважды на русском и английском языке для демонстрации подхода к нумерации страниц.}
% Аннотация из renames.tex

% \abstractRu % УДАЛИТЬ. Повтор иллюстрации переноса текста на вторую страницу



\printTheAbstract % не удалять


\total{mypages}~p., 
\total{myfigures}~figures, 
\total{mytables}~tables,
\total{myappendices}~appendices.

%\noindent
{\MakeUppercase{Keywords: \keywordsEn}.} % Ключевые слова из renames.tex 
	
The subject of the graduate qualification work is <<\thesisTitleEn>>.
	
	
\abstractEn % Аннотация из renames.tex

% \abstractEn % УДАЛИТЬ. Повтор для иллюстрации переноса текста на вторую страницу
	


%% Не менять - Do not modify
\thispagestyle{empty}
%\setcounter{sumPageLast}{\value{page}} %сохранили номер последней страницы Задания
%\setcounter{sumPages}{\value{sumPageLast}-\value{sumPageFirst}}
%sumPageLast \arabic{sumPageLast}
%
%sumPages \arabic{sumPages}
%\restoregeometry % восстанавливаем настройки страницы
%\setcounter{sumPageLast}{\value{page}} %сохранили номер последней страницы Задания
\setcounter{sumPages}{\value{sumPageLast}-\value{sumPageFirst}}
\arabic{sumPageLast}
\arabic{sumPages}
\restoregeometry % восстанавливаем настройки страницы
\pagestyle{plain} % удаляем номер страницы на первой/второй странице Задания
%% Обязательно закомментировать, если получается один лист в задании:
\ifnumequal{\value{sumPages}}{0}{% Если 1 страница в Задании, то ничего не делать.
}{% Иначе
	\setcounter{page}{\value{page}-\value{sumPages}} 	% вычесть значение sumPages при печати более 1 страницы страниц
}%	% настройки - конец