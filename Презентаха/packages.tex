%!TeX root = ./SimplifyJapanese.tex
\usetheme{metropolis}
% \setbeamertemplate{footline}[frame number]{}

% \usepackage{tikz}
% \usetikzlibrary{arrows,automata,positioning}
\usepackage{lastpage}

\usepackage[english, russian]{babel}

\usepackage{tabularx} % таблицы
\renewcommand\tabularxcolumn[1]{m{#1}}% for vertical centering text in X column

% \usepackage{fancybox, fancyhdr}
% \usepackage{ulem, color, tcolorbox, graphicx}
% \usepackage{float, wrapfig, subcaption}
% \usepackage{enumerate} % дополнительный функционал для списков
% \usepackage{multicol} % разделение на несколько колонок
% \usepackage{ifthen} % условия
% % \usepackage{physics} % содержит дифференциальные операторы (\dd, \dv, \pdv и т. д.) и не только

% % вертикальное выравнивание колонок в таблице
% \renewcommand\tabularxcolumn[1]{m{#1}}

% \DeclareMathAlphabet{\mathcal}{OMS}{cmsy}{m}{n}
% \let\mathbb\relax % remove the definition by unicode-math
% \DeclareMathAlphabet{\mathbb}{U}{msb}{m}{n}
% % \setmathfont{mathFont.ttf}
% \setmathrm{mathFont.ttf}
% \setmathfont(Digits,Latin){mathFont.ttf}
% % \setmathfont{Latin Modern Math}

\defaultfontfeatures{Ligatures=TeX}

\setbeamertemplate{footline}[text line]{%
  \parbox{\linewidth}{\scriptsize\vspace*{-10pt}Фурман~В.\,К. «Разработка и исследование системы автоматического упрощения текстов на японском языке»\hfill\small\insertpagenumber/\pageref{LastPage}}}
\setbeamertemplate{navigation symbols}{}

\newcommand{\import}[1]{\input{C:/Users/Ruminat/GoogleDrive/TeX/commands/#1}} % импорт из папки commands

% настройка изображений
\DeclareGraphicsExtensions{.pdf,.png,.jpg,.jpeg}
\graphicspath{{./img/}}

\usepackage{tikz}
\usetikzlibrary{arrows,automata,positioning}
\usepackage{float}

\usepackage{amssymb}
\usepackage{pdfpages}


% \usepackage{CJKutf8}
% \usepackage{CJK,CJKspace,CJKpunct}
\pdfmapline{=unisong@Unicode@ <ipam.ttf}
\usepackage{minted}

\usepackage{xeCJK}
\setCJKmainfont{NotoSerifCJKjp-Regular.otf}
\newcommand{\yubi}[2]{%
  % \ensuremath{\underbrace{\text{#1}}_{\text{#2}}}%
  \ensuremath{\stackrel{\text{#2}}{\text{#1}}}
}

\import{textCommands}            % текстовые команды
% \import{OLD/mathCommands}            % математические команды
% \import{OLD/RUSmathCommands}         % особенности русской типографии
% \import{OLD/linearAlgebraCommands}   % линейная алгебра
% \import{OLD/setsAndDiscreteCommands} % множества и дискретная математика