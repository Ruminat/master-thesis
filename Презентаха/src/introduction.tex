%!TeX root = ../SimplifyJapanese.tex
\begin{frame}[fragile]{Цель и задачи работы}%
  \Large
  Цель
  \normalsize

  Разработка и исследование системы автоматического упрощения текстов на японском языке.

  \Large
  Задачи
  \normalsize

  \begin{itemize}%
    \item Исследование предметной области;
    \item исследование существующих решений и технологий (ИНС);
    \item разработка системы упрощения, а также её улучшение;
    \item сбор и анализ метрик, обзор упрощения предложений разработанной системой.
  \end{itemize}
\end{frame}


\begin{frame}[fragile]{Мотивация}%
  \begin{itemize}%
    \item Японский язык очень непростой (порой и для самих японцев). В основном из-за иероглифов (но не только).
    \item Упрощение текстов "--- это расширение их потенциальных читателей, упрощение понимания смысла текстов.
    \item С системами упрощения на сегодняшний день всё непросто "--- их мало и они закрыты (не выходят за рамки статей).
  \end{itemize}
\end{frame}


\begin{frame}[fragile]{Как упрощать тексты на японском}%
  В японском есть 3 вида письменности:
  \begin{itemize}%
    \item 2 азбуки "--- хирагана (ひらがな) и катакана (カタカナ);
    \item иероглифы (кандзи) (漢字).
  \end{itemize}

  Как можно упрощать тексты на японском:
  \begin{itemize}%
    \item заменять сложные слова (обычно из кандзи) на простые;
    \item менять формальные грамматические конструкции на разговорные;
    \item заменять использование некоторых иероглифов на азбуку.
  \end{itemize}
\end{frame}
